%% 
%% Created in 2018 by Martin Slapak
%% last update:		2020-02-09
%%
%% Based on file for NRP report LaTeX class by Vit Zyka (2008)
%% enhanced for MI-MVI (2018) and tuned for BI-PYT (2020)
%%
%% Compilation:
%% >pdflatex report
%% >bibtex report
%% >pdflatex report
%% >pdflatex report

% arara: xelatex
% arara: bibtex
% arara: xelatex
% arara: xelatex


\documentclass[czech]{template/pyt-report}

\title{Webová aplikace na organizaci fotografií \emph{photoghast}}

\author{David Košťál}
\affiliation{FIT ČVUT}
\email{kostada2@fit.cvut.cz}

\def\file#1{{\tt#1}}

\begin{document}

\maketitle

%%%%%%%%%%%%%%%%%%%%%%%%%%%%%%%%%%%%%%%%%%%%%%%%%%%%%%%%%%%%%%%%%%%%%%%%%%%%%%%%
\section{Úvod}
Cílem této semestrální práce je vytvořit webovou aplikaci, která usnadní správu
knihovny fotografií. Aplikace si nejprve zaindexuje adresář s fotkami, pro které
si uloží metadata z EXIF a dalších zdrojů (mimo jiné datum pořízení, rozměry,
GPS souřadnice) a vygeneruje náhledové obrázky. Všechny tyto údaje jsou uloženy do
databáze (výchozí je SQLite, ale lze použít jiné možnosti). 

Tyto fotografie je poté možné pohodlně prohlížet ve webové aplikaci v několika formách.
Výchozí forma je kontinuální časová osy od nejnovějších k nejstarším obrázkům.
Další zobrazení jsou alba, automaticky vytvořená podsložkami v hlavním adresáři.
Posledním zobrazením jsou místa pořízení, kde jsou automaticky seskupené blízké fotografie
(data pro tuto funkci se získávají s EXIF souřadnic).

Pro přístup k knihovně je potřeba se přihlásit, aplikace disponuje jednoduchým
systémem uživetelů, kde administrátoři mohou přidávat nové uživatele.
%%%%%%%%%%%%%%%%%%%%%%%%%%%%%%%%%%%%%%%%%%%%%%%%%%%%%%%%%%%%%%%%%%%%%%%%%%%%%%%%
\section{Indexace obrázků}
Aby bylo možné obrázky zobrazit v rozhraní, je potřeba je nejdříve zaindexovat
a uložit do databáze. Jako rozhraní pro databázi je použita knihovna SQLAlchemy,
která podporuje mnoho databází, v této aplikaci je ale momentálně použita databáze
SQLite.

Soubor \verb+indexer/indexer.py+ je zodpovědný za indaxci a po spuštění začně
rekurzivně procházet všechny soubory a složky v zadaném adresáři. Pokud narazí
na složku, uloží k všem obrázkům v této složce jméno této složky jako album obrázku.
Z veškerých podporovaných obrázků (JPG, PNG) na které narazí, se pokusí extrahovat
EXIF informace pomocí programu \verb+exiftool+ a zaznamenat je.

Dále se obrázky s metadaty o poloze pokusí umístit do lokace. Nejprve zkusí, zda
již neexistuje blízká lokace (nachází se v určité maximální vzdálenosti),
kam by se obrázek mohl zařadit. Pokud není nalezena, založí se pro obrázek nová lokace.
Pro každou lokaci se dotáže na OpenStreetMap Nominatim API, aby se pro souřadnice
získala užitečná lokace.

V neposlední řadě se také generují zmenšené náhledové obrázky, pomocí Numpy.
Generátor náhledových obrázků spočítá zmenšení, aby se blížilo požadované velikosti
a zároveň se příliš nedeformoval poměr stran. Je použita metoda primitivního 
výběru pixelů bez interpolace.

Pokud přibyly nebo ubyly nějaké obrázky od poslední indexace, vždy se přidají jenom změny.
%%%%%%%%%%%%%%%%%%%%%%%%%%%%%%%%%%%%%%%%%%%%%%%%%%%%%%%%%%%%%%%%%%%%%%%%%%%%%%%%
\section{Webová aplikace}
Webová aplikace, postavená na knihovně Flask, nabídne přihlášeným uživatelům
prohlížení indexované knihovny. Je organizovaná do stránek se režimy organizace fotek.
Má cesty jak s HTML templaty, tak i cesty fungující jako API (pro získání náhledu 
a celé fotografie pomocí ID, kde se buď získá z databáze, nebo načte přímo z disku).
Kromě prohlížení fotek je pro administrátory možné vytvořit nové uživatele, a pro všechny
uživatele je možné si změnit heslo.

Pro práci s hesly je použita knihova bcrypt, která poskytuje bezpečné hashování
a porovnávání hesel. Pro udržování přihlášení slouží knihovna Flask-Login. WTForms
a Flask-WTF je knihovna pro generování formulářů, a chrání je před CSRF útoky.

Webový frontend využívá knihovny Bootstrap pro CSS a Masonry pro mřížku fotek.
Okno s plnou verzí po kliknutí zajišťuje knihovna PhotoSwipe Lightbox.
Celý web je responzivní.
%%%%%%%%%%%%%%%%%%%%%%%%%%%%%%%%%%%%%%%%%%%%%%%%%%%%%%%%%%%%%%%%%%%%%%%%%%%%%%%%
% --- VYSLEDKY
\section{Výsledky}
V aplikaci chybí velké množství funkcí, ale jako základ je bez problému použitelná.
Použití velkou knihovnou fotografií by bez přidání dalších funkcí bylo ale pomalé.
Se zvolenými Pythonovými nástroji se pracuje velmi pohodlně a přidávat funkce není problém.
Největší problémy jsem měl s frontendovou částí webu, obzvlášť s rozložením mřířky fotek.

\begin{figure}[h]
  \centering\leavevmode
  \includegraphics[width=.45\linewidth]{template/img/fit-logo-cz.pdf}\vskip-0.5cm
  \caption{Vliv parametru \emph{Y}}
  \label{fig:par-y}
\end{figure}
\begin{figure}[h]
  \centering\leavevmode
  \includegraphics[width=.45\linewidth]{template/img/fit-logo-en.pdf}\vskip-0.5cm
  \caption{Vliv parametru \emph{X}}
  \label{fig:par-x}
\end{figure}

Jak je z ilustrací \ref{fig:par-y} a \ref{fig:par-x} patrné, není to totéž, protože\ldots


%%%%%%%%%%%%%%%%%%%%%%%%%%%%%%%%%%%%%%%%%%%%%%%%%%%%%%%%%%%%%%%%%%%%%%%%%%%%%%%%
% --- ZAVER
\section{Závěr}
K čemu to bylo/je dobré, jak to půjde využít dále, co by šlo ještě vylepšit\ldots 

%%%%%%%%%%%%%%%%%%%%%%%%%%%%%%%%%%%%%%%%%%%%%%%%%%%%%%%%%%%%%%%%%%%%%%%%%%%%%%%%
% --- Bibliography
\nocite{zizka}
%\bibliographystyle{plain-cz-online}
\bibliography{reference}

\end{document}
