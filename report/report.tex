%% 
%% Created in 2018 by Martin Slapak
%% last update:		2020-02-09
%%
%% Based on file for NRP report LaTeX class by Vit Zyka (2008)
%% enhanced for MI-MVI (2018) and tuned for BI-PYT (2020)
%%
%% Compilation:
%% >pdflatex report
%% >bibtex report
%% >pdflatex report
%% >pdflatex report

% arara: xelatex
% arara: bibtex
% arara: xelatex
% arara: xelatex


\documentclass[czech]{template/pyt-report}

\title{Webová aplikace na organizaci fotografií \emph{photoghast}}

\author{David Košťál}
\affiliation{FIT ČVUT}
\email{kostada2@fit.cvut.cz}

\def\file#1{{\tt#1}}

\begin{document}

\maketitle

%%%%%%%%%%%%%%%%%%%%%%%%%%%%%%%%%%%%%%%%%%%%%%%%%%%%%%%%%%%%%%%%%%%%%%%%%%%%%%%%
\section{Úvod}
Cílem této semestrální práce je vytvořit webovou aplikaci, která usnadní správu
knihovny fotografií. Aplikace si nejprve zaindexuje adresář s fotkami, pro které
si uloží metadata z EXIF a dalších zdrojů (mimo jiné datum pořízení, rozměry,
GPS souřadnice) a vygeneruje náhledové obrázky. Všechny tyto údaje jsou uloženy do
databáze (výchozí je \href{https://www.sqlite.org/index.html}{SQLite},
ale lze použít jiné možnosti). 

Tyto fotografie je poté možné pohodlně prohlížet ve webové aplikaci v několika formách.
Výchozí forma je kontinuální časová osa od nejnovějších k nejstarším obrázkům.
Další zobrazení jsou alba, automaticky vytvořená podsložkami v hlavním adresáři.
Posledním zobrazením jsou místa pořízení, kde jsou automaticky seskupené blízké fotografie
(data pro tuto funkci se získávají z EXIF souřadnic).

Pro přístup k webové knihovně je potřeba se přihlásit, aplikace disponuje jednoduchým
systémem uživatelů, kde administrátoři mohou přidávat nové uživatele.
%%%%%%%%%%%%%%%%%%%%%%%%%%%%%%%%%%%%%%%%%%%%%%%%%%%%%%%%%%%%%%%%%%%%%%%%%%%%%%%%
\section{Indexace obrázků}
Aby bylo možné obrázky zobrazit v rozhraní, je potřeba je nejprve zaindexovat
a uložit do databáze. Jako rozhraní pro databázi je použita knihovna 
\href{https://www.sqlalchemy.org}{SQLAlchemy}, která podporuje mnoho databází,
v této aplikaci je ale momentálně použita databáze SQLite.

Soubor \verb+indexer/indexer.py+ je zodpovědný za indexci a po spuštění začně
rekurzivně procházet všechny soubory a složky v zadaném adresáři. Pokud narazí
na složku, uloží ke všem obrázkům ve složce její jméno jako album obrázku.
Z veškerých podporovaných obrázků (JPG, PNG) na které narazí, se pokusí extrahovat
EXIF informace pomocí programu \href{https://exiftool.org/}{exiftool} a zaznamenat je.

Dále se obrázky s metadaty o poloze pokusí umístit do lokace. Nejprve zkusí, zda
již neexistuje blízká lokace (nachází se v určité maximální vzdálenosti),
kam by se obrázek mohl zařadit. Pokud není nalezena, založí se pro obrázek nová lokace.
Pro každou lokaci se dotáže na \href{https://nominatim.org/}{OpenStreetMap Nominatim API},
aby se pro souřadnice získala užitečná lokace. Experimentováním jsem došel k nastavení,
které neposílá přesné souřadnice z důvodu ochrany soukromí, ale zároveň poskytuje užitečné polohy.

V neposlední řadě se také generují zmenšené náhledové obrázky pomocí 
\href{https://numpy.org/}{Numpy}.
Generátor náhledových obrázků spočítá zmenšení, aby se blížilo požadované velikosti
a zároveň se příliš nedeformoval poměr stran. Je použita metoda primitivního 
výběru pixelů bez interpolace. Indexace podporuje i obrázky s rotací v EXIF metadatch.

Pokud přibyly nebo ubyly nějaké obrázky od poslední indexace, vždy se přidají jenom změny.
%%%%%%%%%%%%%%%%%%%%%%%%%%%%%%%%%%%%%%%%%%%%%%%%%%%%%%%%%%%%%%%%%%%%%%%%%%%%%%%%
\section{Webová aplikace}
Webová součást, postavená na knihovně \href{https://flask.pocoo.org/}{Flask}, nabídne
přihlášeným uživatelům prohlížení indexované knihovny. Je rozdělená do stránek
s režimy organizace fotek.
Má cesty jak s HTML templaty, tak i cesty fungující jako API pro získání dat přes 
identifikátor (například pro získání náhledu fotografie -- ten se získá z databáze
nebo načtení fotky v plném rozlišení -- ta se načte přímo z disku). Kromě prohlížení
fotek je pro administrátory možné vytvořit nové uživatele, a každý uživatel si
může změnit heslo.

Pro práci s hesly je použita knihova \href{https://github.com/pyca/bcrypt/}{bcrypt},
která poskytuje bezpečné hashování a porovnávání hesel.
Pro udržování přihlášení slouží knihovna \href{https://github.com/maxcountryman/flask-login}{Flask-Login}.
\href{https://github.com/wtforms/wtforms}{WTForms} a \href{https://github.com/lepture/flask-wtf}{Flask-WTF}
jsou použité knihovny pro generování formulářů, a jejich ochranou před CSRF útoky.
Sdílené databázové modely z SQLAlchemy mezi indexerem a webem výrazně zjednodušují
práci s databází a údržbu kódu.

Webový frontend využívá knihovny \href{https://getbootstrap.com/}{Bootstrap} pro CSS
a \href{https://masonry.desandro.com/}{Masonry} pro mřížku fotek.
Okno s plnou verzí po kliknutí zajišťuje knihovna PhotoSwipe Lightbox.
Celý web je responzivní.
%%%%%%%%%%%%%%%%%%%%%%%%%%%%%%%%%%%%%%%%%%%%%%%%%%%%%%%%%%%%%%%%%%%%%%%%%%%%%%%%
% --- VYSLEDKY
\section{Výsledky}
V aplikaci chybí velké množství funkcí, ale jako základ je bez problému použitelná, ovšem
použití s velkou knihovnou fotografií by bez přidání dalších optimalizací bylo pomalé.
Se zvolenými Pythonovými nástroji se pracuje velmi pohodlně a přidávat funkce není problém,
největší problémy jsem měl s frontendovou částí webu, obzvlášť s rozložením mřížky fotek.
Další věcí, která nebyla úplně přímočará byly unit testy, z povahy programu není jedoduché
oddělit komponenty, aby se daly testovat.

%%%%%%%%%%%%%%%%%%%%%%%%%%%%%%%%%%%%%%%%%%%%%%%%%%%%%%%%%%%%%%%%%%%%%%%%%%%%%%%%
% --- ZAVER
\section{Závěr}
Myslím si, že photoghast může být dobrou alternativou pro cloudové služby jako Fotky Google,
pro uživatele, kteří mají svoje fotografie uložené lokálně. Aplikace má mnoho potenciálu
pro další vývoj, mimo jiné hledání a využití umělé inteligence na detekci objektů a osob.
%%%%%%%%%%%%%%%%%%%%%%%%%%%%%%%%%%%%%%%%%%%%%%%%%%%%%%%%%%%%%%%%%%%%%%%%%%%%%%%%
% --- Bibliography
% \nocite{zizka}
% \bibliographystyle{plain-cz-online}
% \bibliography{reference}

\end{document}
